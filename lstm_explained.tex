\documentclass[11pt]{article}
\usepackage[utf8]{inputenc}
\usepackage{amsmath}
\usepackage{graphicx}
\usepackage{import}
\usepackage{color}
\usepackage[margin=1in]{geometry}
\usepackage{amsmath}
\usepackage{amsfonts}

\usepackage[backend=biber]{biblatex}
\addbibresource{lstm.bib}

\definecolor{auburn}{rgb}{0.43, 0.21, 0.1}

\title{LSTM: From theory to practice}
\author{Adrian Mihai Iosif, Matei Macri, Tudor Berariu}
\date{January 2016}

\begin{document}

\maketitle

\section{Introduction}

This document is both a resource for understanding the mathematics of LSTM (Section~\ref{sec:math}) and a tutorial for step by step implementation in Torch (Section~\ref{sec:torch}).
The LSTM module was first introduced in \cite{hochreiter1997long} as a solution to the \emph{vanishing gradient} problem that made training vanilla RNNs difficult.

\section{Mathematical foundations}
\label{sec:math}

\paragraph{Conventions.} We propose the following notation for the variables:
\begin{center}
\begin{tabular}{c | l}
    $D$ & memory cell dimension \\
    $N$ & input vector size \\
    \hline
    $g_{\iota}\in \mathbb{R}$  & input gate \\
    $g_{\phi}\in \mathbb{R}$  & forget gate \\
    $g_{\omega}\in \mathbb{R}$  & output gate \\
    \hline
    $\mathbf{z}_{c} \in \mathbb{R}^{D}$ & input vector \\
    $\mathbf{z}_h \in \mathbb{R}^{D}$ & output vector \\
    \hline
    $\mathbf{w}_{\iota} \in \mathbb{R}^{(N + 2D + 1)}$ & input gate parameters \\
    $\mathbf{w}_{\phi} \in \mathbb{R}^{(N + 2D + 1)}$ & forget gate parameters \\
    $\mathbf{w}_{\omega} \in \mathbb{R}^{(N + 2D + 1)}$ & output gate parameters \\
    $\mathbf{W}_{c} \in \mathbb{R}^{(N + D + 1)\times D}$ & input vector parameters \\
    \hline
    $f_{*}$ & activation functions (e.g. logistic) \\
    \hline
    $\mathbf{x} \in \mathbb{R}^{N}$ & inputs \\
    $\mathbf{s} \in \mathbb{R}^{D}$ & the actual memory of the cell
\end{tabular}
\end{center}

We use a simplified notation for the vertical concatenation of two vectors $\mathbf{a}$, $\mathbf{b}$: $\left[\mathbf{a}; \mathbf{b}\right]$ instead of $\left[ \mathbf{a}^{\text{T}} ; \mathbf{b}^{\text{T}} \right]^{\text{T}}$.

\begin{center}
\input{lstm_variables.pdf_tex}
\end{center}

\subsection{The Forward Phase}

\paragraph{Input gate and input value.}

\begin{align}
a_{\iota}^{(t)} &= \underbrace{\mathbf{w}_{\iota}^{\text{T}}}_{1\times(N+2D+1)} \cdot \underbrace{\left[\mathbf{x}^{(t)}; \mathbf{z}_{h}^{(t-1)}; \mathbf{s}^{(t-1)}; 1\right]}_{(N+2D+1)} \\
g_{\iota}^{(t)} &= f_{\iota}\left(a_{\iota}^{(t)}\right)
\end{align}

\begin{align}
\mathbf{a}_{c}^{(t)} &= \underbrace{\mathbf{W}_{c}^{\text{T}}}_{D \times (N+D+1)} \cdot \underbrace{\left[\mathbf{x}^{(t)}; \mathbf{z}_{h}^{(t-1)}; 1\right]}_{(N+D+1)} \\
\mathbf{z}_{c}^{(t)} &= f_{c}\left(\mathbf{a}_{c}^{(t)}\right)
\end{align}

\paragraph{Forget gate.} This is actually a \emph{keep} gate:

\begin{align}
a_{\phi}^{(t)} &= \mathbf{w}_{\phi}^{\text{T}} \cdot \left[\mathbf{x}^{(t)}; \mathbf{z}_{h}^{(t-1)}; \mathbf{s}^{(t-1)}; 1\right] \\
g_{\phi}^{(t)} &= f_{\phi}\left(a_{\phi}^{(t)}\right)
\end{align}

\paragraph{Cell value.}

\begin{equation}
    \mathbf{s}^{(t)} = g_{\phi}^{(t)} \mathbf{s}^{(t-1)} + g_{\iota}^{(t)} \mathbf{z}_{c}^{(t)}
\end{equation}

\paragraph{Output gate and output value.}

\begin{align}
a_{\omega}^{(t)} &= \mathbf{w}_{\omega}^{\text{T}} \cdot \left[\mathbf{x}^{(t)}; \mathbf{z}_{h}^{(t-1)}; \mathbf{s}^{(t)}; 1\right] \\
g_{\omega}^{(t)} &= f_{\omega}\left(a_{\omega}^{(t)}\right) \\
\mathbf{z}_h^{(t)} &= f_h\left(g_{\omega}^{(t)} \mathbf{s}^{(t)}\right)
\end{align}

\subsection{The Backward Phase}

\paragraph{Notations.} In what follows the next notations are used for various partial derivatives:

\begin{center}
\begin{tabular}{c c}
    $\boldsymbol{\delta}_{x}^{(t)} \overset{not.}{=} \displaystyle\frac{\partial E}{\partial z_h^{(t)}}\displaystyle\frac{\partial z_h^{(t)}} {\partial \mathbf{x}^{(t)}}$; &
    $\delta_{zz}^{(t)\rightarrow(t-1)} \overset{not.}{=} \displaystyle\frac{\partial E}{\partial z_h^{(t)}}\displaystyle\frac{\partial z_h^{(t)}} {\partial z_h^{(t-1)}}$; \\[20pt]
\end{tabular}
\begin{tabular}{c c c}
    $\delta_{s}^{(t)} \overset{not.}{=} \displaystyle\frac{\partial E}{\partial s^{(t)}}$; &
    $\delta_{c}^{(t)} \overset{not.}{=} \displaystyle\frac{\partial E}{\partial z_c^{(t)}}$; &
    $\delta_{h}^{(t)} \overset{not.}{=} \displaystyle\frac{\partial E}{\partial z_h^{(t)}}$;  \\[20pt]
    $\delta_{g_{\omega}}^{(t)} \overset{not.}{=} \displaystyle\frac{\partial E}{\partial g_{\omega}^{(t)}}$; &
    $\delta_{g_{\phi}}^{(t)} \overset{not.}{=} \displaystyle\frac{\partial E}{\partial g_{\phi}^{(t)}}$; &
    $\delta_{g_{\iota}}^{(t)} \overset{not.}{=} \displaystyle\frac{\partial E}{\partial g_{\iota}^{(t)}}$ \\[20pt]
    $\boldsymbol{\delta}_{\omega}^{(t)} \overset{not.}{=} \displaystyle\frac{\partial E}{\partial \mathbf{w}_{\omega}}$; &
    $\boldsymbol{\delta}_{\phi}^{(t)} \overset{not.}{=} \displaystyle\frac{\partial E}{\partial \mathbf{w}_{\phi}}$; &
    $\boldsymbol{\delta}_{\iota}^{(t)} \overset{not.}{=} \displaystyle\frac{\partial E}{\partial \mathbf{w}_{\iota}}$ \\[20pt]
    $\boldsymbol{\delta}_{c}^{(t)} \overset{not.}{=} \displaystyle\frac{\partial E}{\partial \mathbf{w}_{c}}$
\end{tabular}
\end{center}

\paragraph{Inner loops.} Before computing the needed gradients, let's take a closer look at $\frac{\partial E}{\partial s^{(t)}}$. This gradient has two components. The first corresponds to the error flowing through $z_h^{(t)}$ and the second corresponds to the inner loops of the LSTM (the connections to $g_{\iota}^{(t+1)}$, $g_{\phi}^{(t+1)}$, and $s^{(t+1)}$).

\begin{equation}
\begin{split}
    \delta_{s}^{(t)} \overset{not.}{=} \displaystyle\frac{\partial E}{\partial s^{(t)}} = &
    {\color{blue} \displaystyle\frac{\partial E}{\partial z_h^{(t)}}
    \displaystyle\frac{\partial z_h^{(t)}}{\partial s^{(t)}}} +
    {\color{auburn}\displaystyle\frac{\partial E}{\partial g_{\phi}^{(t+1)}}
    \displaystyle\frac{\partial g_{\phi}^{(t+1)}}{\partial s^{(t)}} +
    \displaystyle\frac{\partial E}{\partial g_{\iota}^{(t+1)}}
    \displaystyle\frac{\partial g_{\iota}^{(t+1)}}{\partial s^{(t)}} +
    \displaystyle\frac{\partial E}{\partial s^{(t+1)}}
    \displaystyle\frac{\partial s^{(t+1)}}{\partial s^{(t)}}} \\
    = & {\color{blue} \delta_{h}^{(t)} f_{\omega}'\left(g_{\omega}^{(t)}s^{(t)}\right) g_{\omega}^{(t)}} +
        {\color{auburn}\delta_{g_{\phi}}^{(t+1)}f_{\phi}'\left(a_{\phi}^{(t+1)}\right)w_{\phi, s} +
        \delta_{g_{\iota}}^{(t+1)}f_{\iota}'\left(a_{\iota}^{(t+1)}\right)w_{\iota, s} +
        \delta_{s}^{(t+1)}g_{\phi}^{(t+1)}} \\
    = & {\color{blue}\delta^{(t) \rightarrow (t)}_s} +
        {\color{auburn}\delta^{(t+1) \rightarrow (t)}_s}
\end{split}
\end{equation}

\paragraph{Goal.} Given $\delta_h^{(t)}$, and $\delta^{(t+1) \rightarrow (t)}_s$, the following derivatives need to be computed: $\displaystyle \frac{\partial E}{\partial W_{*}}$, $\boldsymbol{\delta}_x^{(t)}$, $\delta_{zz}^{(t)\rightarrow(t-1)}$, and $\delta^{(t) \rightarrow (t-1)}_s$.

\paragraph{Order of computation.}
Gradients need to be computed in the following order: $\delta_{g_{\omega}}^{(t)}$, $\boldsymbol{\delta}_{\omega}^{(t)}$, $\delta_s^{(t) \rightarrow (t)}$, $\delta_s^{(t)}$, $\delta_c^{(t)}$, $\boldsymbol{\delta}_{c}^{(t)}$, $\delta_{g_{\iota}}$, $\boldsymbol{\delta}_{\iota}^{(t)}$, $\delta_{g_{\phi}}$, $\boldsymbol{\delta}_{\phi}^{(t)}$,
$\delta_s^{(t) \rightarrow (t-1)}$, $\boldsymbol{\delta}_x^{(t)}$, $\delta_{zz}^{(t)\rightarrow(t-1)}$.

\begin{align}
    \delta_{g_{\omega}}^{(t)} &\overset{\text{not.}}{=} \displaystyle \frac{\partial E}{\partial g_{\omega}^{(t)}} = \displaystyle \frac{\partial E}{\partial z_{h}^{(t)}} \displaystyle \frac{\partial z_h^{(t)}}{\partial g_{\omega}^{(t)}} = \delta_h^{(t)}f_h'\left(g_{\omega}^{(t)}s^{(t)}\right)s^{(t)} \\
    \boldsymbol{\delta}_{\omega}^{(t)} &\overset{\text{not.}}{=} \displaystyle \frac{\partial E}{\partial \boldsymbol{w}_{\omega}^{(t)}} = \displaystyle \frac{\partial E}{\partial g_{\omega}^{(t)}} \displaystyle \frac{\partial g_{\omega}^{(t)}}{\partial \boldsymbol{w}_{\omega}^{(t)}} = \delta_{g_{\omega}}^{(t)} f_{\omega}'\left(a_{\omega}^{(t)}\right) \cdot \left[\mathbf{x}^{(t)}; z_{h}^{(t-1)}; s^{(t)}; 1\right] \\
    \delta_s^{(t)\rightarrow(t)} &\overset{\text{not.}}{=} \displaystyle \frac{\partial E}{\partial z_h^{(t)}}  \displaystyle \frac{\partial z_h^{(t)}}{s^{(t)}} = \delta_{h}^{(t)} f_{\omega}'\left(g_{\omega}^{(t)}s^{(t)}\right) g_{\omega}^{(t)} \\
    \delta_s^{(t)} &\overset{\text{not.}}{=} \delta^{(t) \rightarrow (t)}_s + \delta^{(t+1) \rightarrow (t)}_s \\
    \delta_c^{(t)} &\overset{\text{not.}}{=} \displaystyle \frac{\partial E}{\partial z_c^{(t)}} = \displaystyle \frac{\partial E}{\partial s^{(t)}}  \displaystyle \frac{\partial s^{(t)}}{z_c^{(t)}} = \delta_s^{(t)} g_i^{(t)} \\
    \boldsymbol{\delta}_{c}^{(t)} &\overset{\text{not.}}{=}  \displaystyle \frac{\partial E}{\partial \boldsymbol{w}_{c}^{(t)}} = \displaystyle \frac{\partial E}{\partial z_c^{(t)}} \displaystyle \frac{\partial z_c^{(t)}} {\partial a_c^{(t)}}
\frac{\partial a_c^{(t)}} {\partial \boldsymbol{w}_c^{(t)}}       
    =  \delta_c^{(t)} f_{c}'\left(a_{c}^{(t)}\right)  \cdot \left[\mathbf{x}^{(t)}; z_{h}^{(t-1)}; 1\right]\\
    \delta_{g_{\iota}}^{(t)} &\overset{\text{not.}}{=} \displaystyle \frac{\partial E}{\partial g_{\iota}^{(t)}} = \displaystyle \frac{\partial E}{\partial s^{(t)}}  \displaystyle \frac{\partial s^{(t)}}{g_{\iota}^{(t)}} = \delta_s^{(t)} z_c^{(t)} \\
    \boldsymbol{\delta}_{\iota}^{(t)} &\overset{\text{not.}}{=} \displaystyle \frac{\partial E}{\partial \boldsymbol{w}_{\iota}^{(t)}} = \displaystyle \frac{\partial E}{\partial g_{\iota}^{(t)}} \displaystyle \frac{\partial g_{\iota}^{(t)}}{\partial \boldsymbol{w}_{\iota}^{(t)}} = \delta_{g_{\phi}}^{(t)} f_{\iota}'\left(a_{\iota}^{(t)}\right) \cdot \left[\mathbf{x}^{(t)}; z_{h}^{(t-1)}; s^{(t-1)}; 1\right] \\
    \delta_{g_{\phi}}^{(t)} &\overset{\text{not.}}{=} \displaystyle \frac{\partial E}{\partial g_{\phi}^{(t)}} = \displaystyle \frac{\partial E}{\partial s^{(t)}}  \displaystyle \frac{\partial s^{(t)}}{g_{\phi}^{(t)}} = \delta_s^{(t)} s^{(t-1)} \\
    \boldsymbol{\delta}_{\phi}^{(t)} &\overset{\text{not.}}{=} \displaystyle \frac{\partial E}{\partial \boldsymbol{w}_{\phi}^{(t)}} = \displaystyle \frac{\partial E}{\partial g_{\phi}^{(t)}} \displaystyle \frac{\partial g_{\phi}^{(t)}}{\partial \boldsymbol{w}_{\phi}^{(t)}} = \delta_{g_{\phi}}^{(t)} f_{\phi}'\left(a_{\phi}^{(t)}\right) \cdot \left[\mathbf{x}^{(t)}; z_{h}^{(t-1)}; s^{(t-1)}; 1\right] \\
    \delta^{(t) \rightarrow (t-1)}_s &= \delta_{g_{\phi}}^{(t)}f_{\phi}'\left(a_{\phi}^{(t)}\right)w_{\phi, s} +
        \delta_{g_{\iota}}^{(t)}f_{\iota}'\left(a_{\iota}^{(t)}\right)w_{\iota, s} +
        \delta_{s}^{(t)}g_{\phi}^{(t)}
\end{align}

\begin{equation}
\begin{split}
    \boldsymbol{\delta}_x^{(t)} &\overset{\text{not.}}{=} \displaystyle\frac{\partial E}{\partial z_h^{(t)}}  \displaystyle\frac{\partial z_h^{(t)}}{\partial \mathbf{x}^{(t)}} \\
    &= \displaystyle\frac{\partial E}{\partial z_h^{(t)}} \left(
            \displaystyle\frac{\partial z_h^{(t)}}{\partial g_{\iota}^{(t)}} \displaystyle\frac{\partial g_{\iota}^{(t)}}{\partial \mathbf{x}^{(t)}} +
            \displaystyle\frac{\partial z_h^{(t)}}{\partial g_{\phi}^{(t)}} \displaystyle\frac{\partial g_{\phi}^{(t)}}{\partial \mathbf{x}^{(t)}} +
            \displaystyle\frac{\partial z_h^{(t)}}{\partial g_{\omega}^{(t)}} \displaystyle\frac{\partial g_{\omega}^{(t)}}{\partial \mathbf{x}^{(t)}} +
            \displaystyle\frac{\partial z_h^{(t)}}{\partial z_{c}^{(t)}} \displaystyle\frac{\partial z_{c}^{(t)}}{\partial \mathbf{x}^{(t)}} \right)\\
            &= \delta_{g_{\iota}}^{(t)} f'_{\iota}\left(a_{\iota}^{(t)}\right) \mathbf{w}_{\iota, x} +
               \delta_{g_{\phi}}^{(t)} f'_{\phi}\left(a_{\phi}^{(t)}\right) \mathbf{w}_{\phi, x} +
               \delta_{g_{\omega}}^{(t)} f'_{\omega}\left(a_{\omega}^{(t)}\right) \mathbf{w}_{\omega, x} +
               \delta_{c}^{(t)} f'_{c}\left(a_{c}^{(t)}\right) \mathbf{w}_{c, x}
\end{split}    
\end{equation}

\begin{equation}
\begin{split}
    \boldsymbol{\delta}_{zz}^{(t)\rightarrow(t-1)} &\overset{\text{not.}}{=} \displaystyle\frac{\partial E}{\partial z_h^{(t)}}  \displaystyle\frac{\partial z_h^{(t)}}{\partial z_h^{(t-1)}} \\
    &= \displaystyle\frac{\partial E}{\partial z_h^{(t)}} \left(
            \displaystyle\frac{\partial z_h^{(t)}}{\partial g_{\iota}^{(t)}} \displaystyle\frac{\partial g_{\iota}^{(t)}}{\partial z_h^{(t-1)}} +
            \displaystyle\frac{\partial z_h^{(t)}}{\partial g_{\phi}^{(t)}} \displaystyle\frac{\partial g_{\phi}^{(t)}}{\partial z_h^{(t-1)}} +
            \displaystyle\frac{\partial z_h^{(t)}}{\partial g_{\omega}^{(t)}} \displaystyle\frac{\partial g_{\omega}^{(t)}}{\partial z_h^{(t-1)}} +
            \displaystyle\frac{\partial z_h^{(t)}}{\partial z_{c}^{(t)}} \displaystyle\frac{\partial z_{c}^{(t)}}{\partial z_h^{(t-1)}} \right)\\
            &= \delta_{g_{\iota}}^{(t)} f'_{\iota}\left(a_{\iota}^{(t)}\right) \mathbf{w}_{\iota, z} +
               \delta_{g_{\phi}}^{(t)} f'_{\phi}\left(a_{\phi}^{(t)}\right) \mathbf{w}_{\phi, z} +
               \delta_{g_{\omega}}^{(t)} f'_{\omega}\left(a_{\omega}^{(t)}\right) \mathbf{w}_{\omega, z} +
               \delta_{c}^{(t)} f'_{c}\left(a_{c}^{(t)}\right) \mathbf{w}_{c, z}
\end{split}    
\end{equation}

\section{Torch Implementation}
\label{sec:torch}

\appendix

\printbibliography

\end{document}
