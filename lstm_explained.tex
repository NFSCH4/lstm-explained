\documentclass[11pt]{article}

%% Packages
\usepackage[utf8]{inputenc}%
\usepackage{amsmath}%
\usepackage{graphicx}%
\usepackage{import}%
\usepackage{color}%
\usepackage[margin=1in]{geometry}%
\usepackage{amsmath}%
\usepackage{amsfonts}%
\usepackage[backend=biber]{biblatex}%

%% Bibliography
\addbibresource{lstm.bib}

%% Make them nicer
\DeclareMathOperator{\diag}{diag}
\definecolor{auburn}{rgb}{0.43, 0.21, 0.1}

%% Details
\title{LSTM: From theory to practice}%
\author{Adrian Mihai Iosif, Matei Macri, Tudor Berariu\footnote{This
    author wrote Section~\ref{sec:math}.}}%
\date{January 2016}%

\begin{document}

\maketitle

\section{Introduction}

This document is both a resource for understanding the mathematics of
LSTM (Section~\ref{sec:math}) and a tutorial for step by step
implementation in Torch (Section~\ref{sec:torch}).

The LSTM module was first introduced in \cite{hochreiter1997long} as a
solution to the \emph{vanishing gradient} problem that made training
vanilla RNNs difficult.

\section{Mathematical foundations}
\label{sec:math}

\subsubsection*{Notations and other conventions}

In this document the following notations are adopted:

\begin{table}[h]
\centering
\begin{tabular}{c | c | l}
  \multicolumn{2}{c |}{$D$} & memory cell dimension \\
  \multicolumn{2}{c |}{$N$} & input vector size \\
  \hline \hline
  $g_{\iota}$ & $\mathbb{R}$  & input gate \\
  $g_{\phi}$ & $\mathbb{R}$  & forget gate \\
  $g_{\omega}$ & $\mathbb{R}$  & output gate \\
  \hline
  $\mathbf{x}$ & $\mathbb{R}^{N}$ & inputs \\
  $\mathbf{z}_{c}$ & $\mathbb{R}^{D}$ & input vector (a better name, maybe?) \\
  $\mathbf{s}$ & $\mathbb{R}^{D}$ & the actual memory of the cell \\
  $\mathbf{z}_h$ & $\mathbb{R}^{D}$ & output vector \\
  \hline
  $\mathbf{w}_{\iota}$ & $\mathbb{R}^{(N + 2D + 1)}$ & input gate parameters \\
  $\mathbf{w}_{\phi}$ & $\mathbb{R}^{(N + 2D + 1)}$ & forget gate parameters \\
  $\mathbf{w}_{\omega}$ & $\mathbb{R}^{(N + 2D + 1)}$
                            & output gate parameters \\
  $\mathbf{W}_{c}$ & $\mathbb{R}^{(N + D + 1)\times D}$
                            & input vector parameters \\
  \hline\hline
  \multicolumn{2}{c |}{$f_{*}$} & activation functions (e.g. logistic) - applied element-wise
\end{tabular}
\end{table}

\paragraph{Vertical concatenation.} We use a simplified notation for
the vertical concatenation of two vectors $\mathbf{a}$, $\mathbf{b}$:
$\left[\mathbf{a}; \mathbf{b}\right]$ instead of
$\left[ \mathbf{a}^{\text{T}} ; \mathbf{b}^{\text{T}}
\right]^{\text{T}}$.

\paragraph{Element-wise multiplication.} We use $\odot$ to denote
element-wise multiplication of equally sized tensors.

\paragraph{Unidimensional tensors.}

In this document we use the notation $\mathbf{v}$ for a column vector
and $\mathbf{v}^{\text{T}}$ whenever a row vector is needed. However,
for various Jacobians we use a notation such as
$\mathbf{j} = \frac{\partial \alpha}{\partial \mathbf{v}}$ for a row
vector (a $1\times size(\mathbf{v})$ Jacobian). Vectors that are not
defined as Jacobians are always columns.

\subsection{The Forward Phase}

In what follows we describe the general formulas used to compute the
output vector of the LSTM cell.

\subsubsection*{The computational graph}

\begin{center}
  \input{lstm_variables.pdf_tex}
\end{center}

\subsubsection*{Input gate and input value}

\begin{align}
  a_{\iota}^{(t)}
  & = \underbrace{\mathbf{w}_{\iota}^{\text{T}}}_{1\times(N+2D+1)}
    \cdot \underbrace{\left[\mathbf{x}^{(t)};
    \mathbf{z}_{h}^{(t-1)}; \mathbf{s}^{(t-1)};
    1\right]}_{(N+2D+1)} \\
  g_{\iota}^{(t)}
  & = f_{\iota}\left(a_{\iota}^{(t)}\right)
\end{align}

\begin{align}
  \mathbf{a}_{c}^{(t)}
  & = \underbrace{\mathbf{W}_{c}^{\text{T}}}_{D \times (N+D+1)}
    \cdot \underbrace{\left[\mathbf{x}^{(t)}; \mathbf{z}_{h}^{(t-1)};
    1\right]}_{(N+D+1)} \\
  \mathbf{z}_{c}^{(t)}
  & = f_{c}\left(\mathbf{a}_{c}^{(t)}\right)
\end{align}

\subsubsection*{Forget gate}

This gate actually acts as a \emph{keep} gate.

\begin{align}
  a_{\phi}^{(t)}
  & = \mathbf{w}_{\phi}^{\text{T}} \cdot
    \left[\mathbf{x}^{(t)}; \mathbf{z}_{h}^{(t-1)}; \mathbf{s}^{(t-1)};
    1\right] \\
  g_{\phi}^{(t)}
  & = f_{\phi}\left(a_{\phi}^{(t)}\right)
\end{align}

\subsubsection*{Cell value}

\begin{equation}
  \mathbf{s}^{(t)} =
  g_{\phi}^{(t)} \mathbf{s}^{(t-1)} + g_{\iota}^{(t)} \mathbf{z}_{c}^{(t)}
\end{equation}

\subsubsection*{Output gate and output value}

\begin{align}
  a_{\omega}^{(t)}
  & = \mathbf{w}_{\omega}^{\text{T}} \cdot
    \left[\mathbf{x}^{(t)}; \mathbf{z}_{h}^{(t-1)}; \mathbf{s}^{(t)};
    1\right] \\
  g_{\omega}^{(t)} & = f_{\omega}\left(a_{\omega}^{(t)}\right) \\
  \mathbf{z}_h^{(t)} & = f_h\left(g_{\omega}^{(t)} \mathbf{s}^{(t)}\right)
\end{align}

\subsection{The Backward Phase}

In this subsection we present the exact form of the partial
derivatives of some error function with respect to the parameters of
the LSTM cell.

\subsubsection*{Notations for various partial derivatives}

% The partial derivative of the error with respect to x

The partial derivatives of the error $E$ with respect to the parameters:

\begin{align}
     \boldsymbol{\delta}_{\omega}^{(t)} \overset{not.}{=}
     \dfrac{\partial E}{\partial \mathbf{w}_{\omega}}
  &&
     \boldsymbol{\delta}_{\phi}^{(t)} \overset{not.}{=}
     \dfrac{\partial E}{\partial \mathbf{w}_{\phi}}
  &&
     \boldsymbol{\delta}_{\iota}^{(t)} \overset{not.}{=}
     \dfrac{\partial E}{\partial \mathbf{w}_{\iota}} \\
  &&
     \boldsymbol{\Delta}_{c}^{(t)} \overset{not.}{=}
     \dfrac{\partial E}{\partial \mathbf{W}_{c}}
  &&
\end{align}

The partial derivatives of the error $E$ with respect to the gates:

\begin{align}
     \delta_{g_{\omega}}^{(t)} \overset{not.}{=}
     \dfrac{\partial E}{\partial g_{\omega}^{(t)}}
  &&
     \delta_{g_{\phi}}^{(t)} \overset{not.}{=}
     \dfrac{\partial E}{\partial g_{\phi}^{(t)}}
  &&
     \delta_{g_{\iota}}^{(t)} \overset{not.}{=}
     \dfrac{\partial E}{\partial g_{\iota}^{(t)}}
\end{align}


The partial derivatives of the error $E$ with respect to
$\mathbf{x}^{(t)}$ and with respect to $\mathbf{z}_h^{(t-1)}$:

\begin{equation}
    \underbrace{\boldsymbol{\delta}_{x}^{(t)}}_{1 \times N} \overset{not.}{=}
    \underbrace{\dfrac{\partial E}{\partial \mathbf{z}_h^{(t)}}}_{1 \times D}
    \underbrace{\dfrac{\partial \mathbf{z}_h^{(t)}}{\partial \mathbf{x}^{(t)}}}
    _{D \times N}
\end{equation}

\begin{equation}
  \underbrace{\boldsymbol{\delta}_{zz}^{(t)\rightarrow(t-1)}}_{1 \times D}
  \overset{not.}{=}
  \underbrace{\dfrac{\partial E}{\partial \mathbf{z}_h^{(t)}}}_{1 \times D}
  \underbrace{\dfrac{\partial \mathbf{z}_h^{(t)}}
    {\partial \mathbf{z}_h^{(t-1)}}}_{D \times D}
\end{equation}

\begin{align}
     \boldsymbol{\delta}_{s}^{(t)} \overset{not.}{=}
     \dfrac{\partial E}{\partial \boldsymbol{s}^{(t)}}
  &&
     \boldsymbol{\delta}_{\mathbf{z}_c}^{(t)} \overset{not.}{=}
     \dfrac{\partial E}{\partial \boldsymbol{z}_c^{(t)}}
  &&
     \boldsymbol{\delta}_{\mathbf{z}_h}^{(t)} \overset{not.}{=}
     \dfrac{\partial E}{\partial \boldsymbol{z}_h^{(t)}}
\end{align}

\subsubsection*{Inner loops}

Before computing the needed gradients, let's take a closer look at
$\frac{\partial E}{\partial \mathbf{s}^{(t)}}$. This gradient has two
components. The first corresponds to the error flowing through
$\mathbf{z}_h^{(t)}$ and the second corresponds to the inner loops of
the LSTM (the connections to $g_{\iota}^{(t+1)}$, $g_{\phi}^{(t+1)}$,
and $\mathbf{s}^{(t+1)}$).

\begin{equation}
  \begin{split}
    \boldsymbol{\delta}_{s}^{(t)} \overset{not.}{=} \dfrac{\partial
      E}{\partial \boldsymbol{s}^{(t)}} = & {\color{blue}
      \dfrac{\partial E}{\partial \boldsymbol{z}_h^{(t)}}
      \dfrac{\partial \mathbf{z}_h^{(t)}}{\partial
        \boldsymbol{s}^{(t)}}} + {\color{auburn}\dfrac{\partial
        E}{\partial g_{\phi}^{(t+1)}} \dfrac{\partial
        g_{\phi}^{(t+1)}}{\partial \boldsymbol{s}^{(t)}} +
      \dfrac{\partial E}{\partial g_{\iota}^{(t+1)}} \dfrac{\partial
        g_{\iota}^{(t+1)}}{\partial \boldsymbol{s}^{(t)}} +
      \dfrac{\partial E}{\partial \boldsymbol{s}^{(t+1)}}
      \dfrac{\partial \boldsymbol{s}^{(t+1)}}{\partial
        \boldsymbol{s}^{(t)}}}
    \\
    = & {\color{blue}
      \dfrac{\partial E}{\partial \boldsymbol{z}_h^{(t)}}
      \dfrac{\partial \boldsymbol{z}_h^{(t)}}{\partial
        \left(g_w^{(t)} \boldsymbol{s}^{(t)} \right)}
      \dfrac{\partial
        \left(g_w^{(t)} \boldsymbol{s}^{(t)} \right)}
      {\partial \boldsymbol{s}^{(t)}}
    } \\ + & {\color{auburn}\dfrac{\partial
      E}{\partial g_{\phi}^{(t+1)}} \dfrac{\partial
      g_{\phi}^{(t+1)}}{\partial \boldsymbol{s}^{(t)}} +
    \dfrac{\partial E}{\partial g_{\iota}^{(t+1)}} \dfrac{\partial
      g_{\iota}^{(t+1)}}{\partial \boldsymbol{s}^{(t)}} +
    \dfrac{\partial E}{\partial \boldsymbol{s}^{(t+1)}}
    \dfrac{\partial \boldsymbol{s}^{(t+1)}}{\partial
      \boldsymbol{s}^{(t)}}}
  \\
  = & {\color{blue}
    \underbrace{\boldsymbol{\delta}_{\mathbf{z}_h}^{(t)}}_{1\times D}
    \underbrace{\diag\left(f_{h}'\left(g_{\omega}^{(t)}\mathbf{s}^{(t)}\right)\right)}_{D\times D}
    \underbrace{\left( g_{\omega}^{(t)} \mathbf{I}_{D} + f'_{\omega}\left(a_{\omega} \right) \boldsymbol{s}^{(t)}\mathbf{w}^{\text{T}}_{\omega, s} \right) }_{D \times D}} + \\
     & +
     {\color{auburn}\delta_{g_{\phi}}^{(t+1)}f_{\phi}'\left(a_{\phi}^{(t+1)}\right)\mathbf{w}_{\phi,
         s}^{\text{T}} +
       \delta_{g_{\iota}}^{(t+1)}f_{\iota}'\left(a_{\iota}^{(t+1)}\right)\mathbf{w}_{\iota,
         s}^{\text{T}} +
       g_{\phi}^{(t+1)}{\boldsymbol{\delta}_{s}^{(t+1)}}^{\text{T}}} \\
     = & {\color{blue}\boldsymbol{\delta}^{(t) \rightarrow (t)}_s} +
     {\color{auburn}\boldsymbol{\delta}^{(t+1) \rightarrow (t)}_s}
  \end{split}
\end{equation}

\begin{equation}
   {\color{blue}\boldsymbol{\delta}_s^{(t)\rightarrow(t+1)}[i] =
  \displaystyle\sum_{j=1}
      \dfrac{\partial E}{\partial \boldsymbol{z}_h^{(t)}[j]}
      \dfrac{\partial \mathbf{z}_h^{(t)}[j]}{\partial
        \boldsymbol{s}^{(t)}[i]}
      = \displaystyle\sum_{j=1}
      \dfrac{\partial E}{\partial \boldsymbol{z}_h^{(t)}[j]}
      f_{h}'\left(g_{\omega}^{(t)}\mathbf{s}^{(t)}[i]\right)
      \left( g_{\omega} \delta_{i,j} + \mathbf{s}^{(t)}[i]f'_{\omega}\left(a_{\omega}\right)\mathbf{w}_{\omega,s}[j] \right)}
\end{equation}


\subsubsection*{The objective of the backward phase}

Given $\boldsymbol{\delta}_{\mathbf{z}_h}^{(t)}$, and
$\boldsymbol{\delta}^{(t+1) \rightarrow (t)}_s$, the following
derivatives need to be computed: $\boldsymbol{\delta}_{\omega}^{(t)}$,
$\boldsymbol{\delta}_{\phi}^{(t)}$,
$\boldsymbol{\delta}_{\iota}^{(t)}$, $\boldsymbol{\Delta}_{c}^{(t)}$
(for optimization purposes) $\boldsymbol{\delta}_x^{(t)}$,
$\boldsymbol{\delta}_{zz}^{(t)\rightarrow(t-1)}$, and
$\boldsymbol{\delta}^{(t) \rightarrow (t-1)}_s$ (in order to compute
other gradients back in the architecture).

\subsubsection*{Order of computation}
Gradients need to be computed in the following order:
$\delta_{g_{\omega}}^{(t)}$, $\boldsymbol{\delta}_{\omega}^{(t)}$,
$\boldsymbol{\delta}_\mathbf{s}^{(t) \rightarrow (t)}$, $\boldsymbol{\delta}_{\mathbf{s}}^{(t)}$, $\boldsymbol{\delta}_{\mathbf{z}_c}^{(t)}$,
$\boldsymbol{\Delta}_{c}^{(t)}$, $\delta_{g_{\iota}}$,
$\boldsymbol{\delta}_{\iota}^{(t)}$, $\delta_{g_{\phi}}$,
$\boldsymbol{\delta}_{\phi}^{(t)}$,
$\boldsymbol{\delta}_s^{(t) \rightarrow (t-1)}$, $\boldsymbol{\delta}_x^{(t)}$,
$\boldsymbol{\delta}_{zz}^{(t)\rightarrow(t-1)}$.

\subsubsection*{The gradients with respect to the output gate and its parameters}

\begin{align}
  \delta_{g_{\omega}}^{(t)} &\overset{\text{not.}}{=} \dfrac{\partial E}{\partial g_{\omega}^{(t)}} = \dfrac{\partial E}{\partial \mathbf{z}_{h}^{(t)}} \dfrac{\partial \mathbf{z}_h^{(t)}}{\partial g_{\omega}^{(t)}} = \boldsymbol{\delta}_h^{(t)}\left(f_h'\left(g_{\omega}^{(t)}\mathbf{s}^{(t)}\right)\odot\mathbf{s}^{(t)}\right) \\
  \boldsymbol{\delta}_{\omega}^{(t)} &\overset{\text{not.}}{=} \dfrac{\partial E}{\partial \boldsymbol{w}_{\omega}^{(t)}} = \dfrac{\partial E}{\partial g_{\omega}^{(t)}} \dfrac{\partial g_{\omega}^{(t)}}{\partial \boldsymbol{w}_{\omega}^{(t)}} = \delta_{g_{\omega}}^{(t)} f_{\omega}'\left(a_{\omega}^{(t)}\right) \left[\mathbf{x}^{(t)}; \mathbf{z}_{h}^{(t-1)}; \mathbf{s}^{(t)}; 1\right]
\end{align}

\subsubsection*{The gradients with respect to the memory cell}

\begin{align}
  \boldsymbol{\delta}_s^{(t)\rightarrow(t)}
  & \overset{\text{not.}}{=}
    \dfrac{\partial E}{\partial \boldsymbol{z}_h^{(t)}}
    \dfrac{\partial \boldsymbol{z}_h^{(t)}}{\boldsymbol{s}^{(t)}} =
    \boldsymbol{\delta}_{h}^{(t)} \diag\left(f_{\omega}'\left(g_{\omega}^{(t)}\boldsymbol{s}^{(t)}\right)\right) g_{\omega}^{(t)} \\
  & = g_{\omega}^{(t)}\boldsymbol{\delta}_h^{(t)} \odot f_{\omega}'\left(g_{\omega}^{(t)}\boldsymbol{s}^{(t)}\right)^{\text{T}} \\
  \boldsymbol{\delta}_s^{(t)} &\overset{\text{not.}}{=} \boldsymbol{\delta}^{(t) \rightarrow (t)}_s + \boldsymbol{\delta}^{(t+1) \rightarrow (t)}_s
\end{align}

\subsubsection*{The gradients with respect to the input value and its
  parameters}

\begin{align}
  \boldsymbol{\delta}_{\boldsymbol{z}_c}^{(t)}
  & \overset{\text{not.}}{=}
    \dfrac{\partial E}{\partial \boldsymbol{z}_c^{(t)}} =
    \dfrac{\partial E}{\partial \boldsymbol{s}^{(t)}}
    \dfrac{\partial \boldsymbol{s}^{(t)}}{\partial \boldsymbol{z}_c^{(t)}} =
    \boldsymbol{\delta}_s^{(t)} \left(g_i^{(t)}\mathbf{I}_{D}\right)  =
     g_i^{(t)} \boldsymbol{\delta}_s^{(t)}
  \\
  \boldsymbol{\Delta}_{c}^{(t)}
  & \overset{\text{not.}}{=}
    \dfrac{\partial E}{\partial \boldsymbol{W}_{c}^{(t)}} =
    \dfrac{\partial E}{\partial \boldsymbol{z}_c^{(t)}}
    \dfrac{\partial \boldsymbol{z}_c^{(t)}}{\partial \boldsymbol{a}_c^{(t)}}
    \dfrac{\partial \boldsymbol{a}_c^{(t)}}{\partial \boldsymbol{W}_c^{(t)}} =
    \underbrace{\left({\boldsymbol{\delta}_{\mathbf{z}_c}^{(t)}}^{\text{T}} \odot f'_c\left(\mathbf{a}_c\right)\right)}_{D\times 1} \underbrace{\left[{\mathbf{x}^{(t)}}^{\text{T}}; {\mathbf{s}^{(t-1)}}^{\text{T}}; 1\right]}_{1 \times (N+D+1)}
\end{align}

\subsubsection*{The gradients with respect to the input gate and its parameters}

\begin{align}
  \delta_{g_{\iota}}^{(t)}
  & \overset{\text{not.}}{=}
    \dfrac{\partial E}{\partial g_{\iota}^{(t)}} =
    \dfrac{\partial E}{\partial \mathbf{s}^{(t)}}
    \dfrac{\partial \mathbf{s}^{(t)}}{\partial g_{\iota}^{(t)}} =
    \boldsymbol{\delta}_s^{(t)} \mathbf{z}_c^{(t)}
  \\
  \boldsymbol{\delta}_{\iota}^{(t)} &\overset{\text{not.}}{=} \dfrac{\partial E}{\partial \boldsymbol{w}_{\iota}^{(t)}} = \dfrac{\partial E}{\partial g_{\iota}^{(t)}} \dfrac{\partial g_{\iota}^{(t)}}{\partial \boldsymbol{w}_{\iota}^{(t)}} = \delta_{g_{\iota}}^{(t)} f_{\iota}'\left(a_{\iota}^{(t)}\right) \left[{\mathbf{x}^{(t)}}^{\text{T}}; {\mathbf{z}_{h}^{(t-1)}}^{\text{T}}; {\mathbf{s}^{(t-1)}}^{\text{T}}; 1\right]
\end{align}

\subsubsection*{The gradients with respect to the forget gate and its parameters}

\begin{align}
  \delta_{g_{\phi}}^{(t)}
  & \overset{\text{not.}}{=}
    \dfrac{\partial E}{\partial g_{\phi}^{(t)}} =
    \dfrac{\partial E}{\partial \mathbf{s}^{(t)}}
    \dfrac{\partial \mathbf{s}^{(t)}}{\partial g_{\phi}^{(t)}} =
    \boldsymbol{\delta}_s^{(t)} \mathbf{s}^{(t-1)}
  \\
  \boldsymbol{\delta}_{\phi}^{(t)} &\overset{\text{not.}}{=} \dfrac{\partial E}{\partial \boldsymbol{w}_{\phi}^{(t)}} = \dfrac{\partial E}{\partial g_{\phi}^{(t)}} \dfrac{\partial g_{\phi}^{(t)}}{\partial \boldsymbol{w}_{\phi}^{(t)}} = \delta_{g_{\phi}}^{(t)} f_{\phi}'\left(a_{\phi}^{(t)}\right) \left[{\mathbf{x}^{(t)}}^{\text{T}}; {\mathbf{z}_{h}^{(t-1)}}^{\text{T}}; {\mathbf{s}^{(t-1)}}^{\text{T}}; 1\right]
\end{align}

\subsubsection*{The gradients with respect to incoming values}

\begin{equation}
  \label{eq:grad_s_t_t_minus_1}
  \boldsymbol{\delta}^{(t) \rightarrow (t-1)}_s = 
  \delta_{g_{\phi}}^{(t)}f_{\phi}'\left(a_{\phi}^{(t)}\right)\mathbf{w}_{\phi,
    s}^{\text{T}} +
  \delta_{g_{\iota}}^{(t)}f_{\iota}'\left(a_{\iota}^{(t)}\right)\mathbf{w}_{\iota,
    s}^{\text{T}} +
  g_{\phi}^{(t)}{\boldsymbol{\delta}_{s}^{(t)}}^{\text{T}}
\end{equation}

\begin{equation}
  \begin{split}
    \boldsymbol{\delta}_x^{(t)}
    & \overset{\text{not.}}{=}
    \dfrac{\partial E}{\partial \mathbf{z}_h^{(t)}} \dfrac{\partial
      \mathbf{z}_h^{(t)}}{\partial \mathbf{x}^{(t)}}
    \\
    & =
    \dfrac{\partial E}{\partial \mathbf{z}_h^{(t)}}
    \left(
      \dfrac{\partial \mathbf{z}_h^{(t)}}{\partial g_{\iota}^{(t)}}
      \dfrac{\partial g_{\iota}^{(t)}}{\partial \mathbf{x}^{(t)}}
      +
      \dfrac{\partial \mathbf{z}_h^{(t)}}{\partial g_{\phi}^{(t)}}
      \dfrac{\partial g_{\phi}^{(t)}}{\partial \mathbf{x}^{(t)}}
      +
      \dfrac{\partial \mathbf{z}_h^{(t)}}{\partial g_{\omega}^{(t)}}
      \dfrac{\partial g_{\omega}^{(t)}}{\partial \mathbf{x}^{(t)}}
      +
      \dfrac{\partial \mathbf{z}_h^{(t)}}{\partial \mathbf{z}_{c}^{(t)}}
      \dfrac{\partial \mathbf{z}_{c}^{(t)}}{\partial \mathbf{x}^{(t)}}
    \right)
    \\
    & =
    \delta_{g_{\iota}}^{(t)} f'_{\iota}\left(a_{\iota}^{(t)}\right)
    \mathbf{w}_{\iota, x}^{\text{T}} + \delta_{g_{\phi}}^{(t)}
    f'_{\phi}\left(a_{\phi}^{(t)}\right) \mathbf{w}_{\phi, x}^{\text{T}} +
    \delta_{g_{\omega}}^{(t)} f'_{\omega}\left(a_{\omega}^{(t)}\right)
    \mathbf{w}_{\omega, x}^{\text{T}} + \underbrace{\left(\boldsymbol{\delta}_{\mathbf{z}_c}^{(t)} \odot
        {f'_{c}\left(\mathbf{a}_{c}^{(t)}\right)}^{\text{T}}\right)}_{1 \times D} \underbrace{\mathbf{W}_{c, x}^{\text{T}}}_{D \times N}
  \end{split}
\end{equation}

\begin{equation}
  \begin{split}
    \boldsymbol{\delta}_{zz}^{(t)\rightarrow(t-1)}
    \overset{\text{not.}}{=} &
    \dfrac{\partial E}{\partial \mathbf{z}_h^{(t)}} \dfrac{\partial
      \mathbf{z}_h^{(t)}}{\partial \mathbf{z}_{h}^{(t-1)}}
    \\
     = &
    \dfrac{\partial E}{\partial \mathbf{z}_h^{(t)}}
    \left(
      \dfrac{\partial \mathbf{z}_h^{(t)}}{\partial g_{\iota}^{(t)}}
      \dfrac{\partial g_{\iota}^{(t)}}{\partial \mathbf{z}_h^{(t-1)}}
      +
      \dfrac{\partial \mathbf{z}_h^{(t)}}{\partial g_{\phi}^{(t)}}
      \dfrac{\partial g_{\phi}^{(t)}}{\partial \mathbf{z}_h^{(t-1)}}
      +
      \dfrac{\partial \mathbf{z}_h^{(t)}}{\partial g_{\omega}^{(t)}}
      \dfrac{\partial g_{\omega}^{(t)}}{\partial \mathbf{z}_h^{(t-1)}}
      +
      \dfrac{\partial \mathbf{z}_h^{(t)}}{\partial \mathbf{z}_{c}^{(t)}}
      \dfrac{\partial \mathbf{z}_{c}^{(t)}}{\partial \mathbf{z}_h^{(t-1)}}
    \right)
    \\
    = &
    \delta_{g_{\iota}}^{(t)} f'_{\iota}\left(a_{\iota}^{(t)}\right)
    \mathbf{w}_{\iota, z}^{\text{T}} + \delta_{g_{\phi}}^{(t)}
    f'_{\phi}\left(a_{\phi}^{(t)}\right) \mathbf{w}_{\phi, z}^{\text{T}} +
    \delta_{g_{\omega}}^{(t)} f'_{\omega}\left(a_{\omega}^{(t)}\right)
    \mathbf{w}_{\omega, z}^{\text{T}} + 
    \\ & + \underbrace{\left(\boldsymbol{\delta}_{\mathbf{z}_c}^{(t)} \odot
        {f'_{c}\left(\mathbf{a}_{c}^{(t)}\right)}^{\text{T}}\right)}_{1 \times D} \underbrace{\mathbf{W}_{c, z}^{\text{T}}}_{D \times D}
  \end{split}
\end{equation}

\section{Torch Implementation}
\label{sec:torch}

\textbf{TODO}

\appendix

\printbibliography

\end{document}
